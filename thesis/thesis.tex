\documentclass[english,letterpaper,12pt]{article}
%\usepackage[margin=1in]{geometry}

%Encoding, fonts, and language
\usepackage[utf8]{inputenc}
\DeclareUnicodeCharacter{2212}{\textminus}
\usepackage[T1]{fontenc}
\usepackage[pdftex,
            pdfauthor={Max Veit},
            pdftitle={Simulation of Chemical Regulatory Networks},
            pdfkeywords={stochastic simulation regulatory networks}]{hyperref}
\usepackage{fouriernc}
\usepackage{tgschola}
\usepackage{babel}

%Text-level formatting
\usepackage{color}
\usepackage{setspace}
\usepackage{multicol}
\frenchspacing

%Mathematics and symbols
\usepackage{amsmath}
%\usepackage{amssymb}
\usepackage{amsthm}
\usepackage{xfrac}
\usepackage{gensymb}
\usepackage{textcomp}
\usepackage{mathtools}

%High-level formatting
\usepackage[square,numbers]{natbib}
%\numberwithin{equation}{section}

%Figures
\usepackage{graphicx}
\usepackage{float}
\usepackage{subcaption}
%\usepackage[font=scriptsize, it]{caption}
\usepackage{placeins}

%Gnuplot vector images
\usepackage{pgf}
\usepackage{pgfplots}
\usepackage{tikz}
\usepackage{gnuplot-lua-tikz}

% Source code listings
\usepackage{listings}
\usepackage{algorithmic}

% Custom math commands, other shortcuts
\newcommand{\unit}[1]{\;\mathrm{#1}}
\newcommand{\unitfr}[2]{\;\frac{\mathrm{#1}}{\mathrm{#2}}}
\newcommand{\tenexp}[1]{\times10^{#1}}
\newcommand{\dee}{\;\mathrm{d}}
\let\oldvec\vec
\renewcommand{\vec}[1]{\ensuremath{\mathbf{#1}}}
\newcommand{\evec}[1]{\ensuremath{\vec{e}_{#1}}} % standard basis vector
\newcommand{\norm}[2]{\ensuremath{\|#1\|_{#2}}}
\newcommand{\bignorm}[2]{\ensuremath{\left\|#1\right\|_{#2}}}
\newcommand{\infnorm}[1]{\ensuremath{\|#1\|_\infty}}
\newcommand{\reals}{\ensuremath{\mathbb{R}}}
% Physics Domain-Specific
\newcommand{\kB}{\ensuremath{k_\mathrm{B}}}
% General Shortcuts
\newcommand{\figref}[1]{Figure~\ref{#1}}
\newcommand{\secref}[1]{Section~\ref{#1}}

\usepackage{lipsum}

% Headers and Footers
\usepackage{fancyhdr}
\pagestyle{fancy}
\lhead{Max Veit\\University of Minnesota}
\rhead{Simulation of Chemical\\Regulatory Networks}
\chead[]{}
\cfoot{\thepage}
\headheight = 15pt

\begin{document}
\title{Stochastic Simulation of Chemical Regulatory Networks with Delayed Reactions}
\author{Max Veit}
\date{5 May 2014}

\maketitle

\begin{doublespacing}

\lipsum[1-2]

\begin{equation}
    \langle A \rangle = \frac{1}{Z} \int_\Omega A(\vec{p}, \vec{q}) \exp\left(\frac{-\varepsilon(\vec{p}, \vec{q})}{\kB T}\right) \dee \vec{p} \dee \vec{q}
\end{equation}
where
\begin{equation}
    Z = \int_\Omega e^{-\varepsilon(\vec{p}, \vec{q}) / \kB T}\dee \vec{p} \dee \vec{q}.
\end{equation}

The quantity $Z$, usually called the \textbf{partition function}, is sometimes also written as
\begin{equation}
    Z = \sum_{s \in \Omega} \exp\left( \frac{-\varepsilon_s}{\kB T} \right)
    \label{eq:partfun-discrete}
\end{equation}
in the case where the phase space is discrete.

Here, $\Omega = \mathbb{R}^{6N}$, where $N$ is the number of particles. The omega represents the probability space, and \emph{does not} stand for anything related to solid angle.

I cite \cite{bistable-modeling}, \cite{we-exact}, \cite{we-chemkin}, and \cite{delay-oscillations} as well as \cite{gillespie-ssa}.

The IPython interactive interpreter and notebook (\cite{PER-GRA:2007}) has been of great use throughout this project.

\lipsum[3-4]

\end{doublespacing}

\bibliographystyle{apsrev}
\bibliography{citations}

\end{document}

