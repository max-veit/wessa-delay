\documentclass[english,letterpaper,12pt]{article}
%\usepackage[margin=1in]{geometry}

%Encoding, fonts, and language
\usepackage[utf8]{inputenc}
\DeclareUnicodeCharacter{2212}{\textminus}
\usepackage[T1]{fontenc}
\usepackage[pdftex,
            pdfauthor={Max Veit},
            pdftitle={Simulation of Genetic Regulatory Networks},
            pdfkeywords={stochastic simulation genetic regulatory networks epigenetics}]{hyperref}
\usepackage{fouriernc}
\usepackage{tgschola}
\usepackage{babel}

%Text-level formatting
\usepackage{color}
\usepackage{setspace}
\usepackage{multicol}
\frenchspacing

%Mathematics and symbols
\usepackage{amsmath}
\usepackage{amssymb}
\usepackage{amsthm}
\usepackage{xfrac}
\usepackage{gensymb}
\usepackage{textcomp}
\usepackage{mathtools}
\usepackage{siunitx}

%High-level formatting
\usepackage[square,numbers]{natbib}
\sisetup{per-mode=symbol-or-fraction}
%\numberwithin{equation}{section}

%Figures
\usepackage{graphicx}
\usepackage{float}
\usepackage{subcaption}
%\usepackage[font=scriptsize, it]{caption}
\usepackage{placeins}

%Gnuplot vector images
\usepackage{pgf}
\usepackage{pgfplots}
\usepackage{tikz}
\usepackage{gnuplot-lua-tikz}

% Source code listings
\usepackage{listings}
\usepackage{algorithmic}

% Custom math commands, other shortcuts
\newcommand{\tenexp}[1]{\times10^{#1}}
\newcommand{\dee}{\;\mathrm{d}}
\let\oldvec\vec
\renewcommand{\vec}[1]{\ensuremath{\mathbf{#1}}}
\newcommand{\evec}[1]{\ensuremath{\vec{e}_{#1}}} % standard basis vector
\newcommand{\norm}[2]{\ensuremath{\|#1\|_{#2}}}
\newcommand{\bignorm}[2]{\ensuremath{\left\|#1\right\|_{#2}}}
\newcommand{\infnorm}[1]{\ensuremath{\|#1\|_\infty}}
\newcommand{\reals}{\ensuremath{\mathbb{R}}}
\DeclareMathOperator{\Prob}{P}
% Physics Domain-Specific
\newcommand{\kB}{\ensuremath{k_\mathrm{B}}}
% General Shortcuts
\newcommand{\figref}[1]{Figure~\ref{#1}}
\newcommand{\secref}[1]{Section~\ref{#1}}

\usepackage{lipsum}

% Headers and Footers
\usepackage{fancyhdr}
\pagestyle{fancy}
\lhead{Max Veit\\University of Minnesota}
\rhead{Simulation of Genetic\\Regulatory Networks}
\chead[]{}
\cfoot{\thepage}
\headheight = 15pt

\begin{document}
\title{Stochastic Simulation of Genetic Regulatory Networks with Delayed Reactions}
\author{Max Veit}
\date{5 May 2014}

\maketitle

\begin{doublespacing}

\section{Introduction} % (fold)
\label{sec:introduction}

%TODO Background references look a little thin - maybe add some more from browsing around?
Recent research in biological physics~\cite{ecoli-decision} indicates that the behavior and internal workings of a living cell is much more rich and complex than the bare instructions coded into its DNA would suggest. For example, an individual section of DNA can be turned off when a repressor protein binds to the beginning of that section. Such mechanisms provide cells a way to change the expression of their DNA, i.e. to control which proteins are produced from their genes and in what amounts, an ability known as (genetic) transcriptional regulation. This ability allows genetically identical cells to adapt their behavior to different environments or to differentiate into different types of cells as in the development of a multicellular organism. Developing models for the selective expression of genes is key to understanding how cells perform their daily functions and respond to their enviroments.

%TODO Ref on biological circuitry would be helpful
Another interesting feature of transcriptional regulation is the presence of feedback. Real cells have many \textit{(how many?)} genes that can each individually be switched on or off. This allows genetic switches to interact in complex ways, forming a genetic regulator network. An example would be two regions of DNA that each code for proteins that suppress the other region, although much more complex feedback mechanisms are found in real cells as well as synthesized in laboratories \textit{(examples?)}. Such mechanisms have the potential to be harnessed as a form of biological circuitry - computation done with chemical reactions instead of electricity~\cite{bio-circuits}. Basic components, such as oscillators, have already been synthesized~\cite{synth-osc}. Biological circuitry promises to deliver a level of control over cells that would allow bacteria to be harnessed for producing chemicals or fulfilling other useful roles in the body. Advancement of this field, however, depends on a better understanding of genetic networks and their behavior.

Many efforts have been made to theoretically model genetic regulatory networks \textit{(refs? examples?)}. Strategies for modeling such networks must also account for the stochasticity resulting from the presence of thermal fluctuations coupled with the small size of cells. Experiments~\cite{ecoli-decision} show that this stochasticity has a major influence on the function of these networks.

One method for modeling genetic regulatory networks is a technique known as system size expansion. This analytical technique captures some of the stochastic character of real genetic regulatory networks, but has some limitations. \textit{(which? where do results differ from real cells? Expand.)}

The method explored in this work is to directly simulate the sequence of chemical reactions occurring in the cell using a Monte Carlo algorithm. The algorithm is known as the Gillespie stochastic simulation algorithm (SSA) and has seen use before in the context of genetic networks \textit{(refs?)}. The aim of this work is to extend the SSA to make it more practical for simulating real-world (natural or synthetic) genetic regulatory networks in order to analyze their behavior.

% section Introduction (end)

\section{Stochastic Chemical Kinetics} % (fold)
\label{sec:chemkin}

%     Reaction rate equations (not necessary...)
%     Only really useful as contrast/strawman

% Chemical kinetics and the Master equation
% Explicit form of the Master equation 

% Motivation: Non-Markovian (i.e. delayed) dynamics


% section chemkin (end)

\section{Methodology} % (fold)
\label{sec:methodology}

\subsection{Gillespie SSA} % (fold)
\label{sub:gillespie-ssa}

% subsection gillespie-ssa (end)

\subsection{Extension to Non-Markovian Dynamics} % (fold)
\label{sub:non-markovian}

% Worth some discussion - Bratsun paper does it differently. Why do it my way?

% subsection non-markovian (end)

\subsection{Weighted-Ensemble Resampling} % (fold)
\label{sub:we-resampling-intro}

% subsection we-resampling-intro (end)

\subsection{Weighted Ensemble With Delays} % (fold)
\label{sub:we-delays}

% subsection we-delays (end)

\subsection{Verification Test Cases} % (fold)
\label{sub:verification}

% System for which analytical distribution is known

% Simple production-degradation system

% subsection verification (end)

% section Methodology (end)

\section{Model Reaction Systems} % (fold)
\label{sec:model-systems}
% TBD
% section model-systems (end)

\section{Results} % (fold)
\label{sec:results}
% TODO
% Some results available - talk about testing of WE, investigation of delayed-deg system
% section resultsResults (end)

\section{Conclusions and Future Work} % (fold)
\label{sec:conclusions}

% Future Work
\subsection{Modeling Diffusion and Crowded Environments} % (fold)
\label{sub:diffusion-crowded}

% subsection diffusion-crowded (end)

%     Crowded environments

% section conclusions (end)

\end{doublespacing}

\appendix
\section{Preview Material} % (fold)
\label{sec:preview-material}


\lipsum[1-2]



\begin{equation}
    \langle A \rangle = \frac{1}{Z} \int_\Omega A(\vec{p}, \vec{q}) \exp\left(\frac{-\varepsilon(\vec{p}, \vec{q})}{\kB T}\right) \dee \vec{p} \dee \vec{q}
\end{equation}
where
\begin{equation}
    Z = \int_\Omega e^{-\varepsilon(\vec{p}, \vec{q}) / \kB T}\dee \vec{p} \dee \vec{q}.
\end{equation}

The quantity $Z$, usually called the \textbf{partition function}, is sometimes also written as
\begin{equation}
    Z = \sum_{s \in \Omega} \exp\left( \frac{-\varepsilon_s}{\kB T} \right)
    \label{eq:partfun-discrete}
\end{equation}
in the case where the phase space is discrete.

Here, $\Omega = \mathbb{R}^{6N}$, where $N$ is the number of particles. The omega represents the probability space, and \emph{does not} stand for anything related to solid angle.

The simple production-degradation equilibrium system with a single species $X$ can be written as the following set of reaction equations:
\begin{align}
    \varnothing \xrightarrow{k_+} X \\
    X \xrightarrow {k_-} \varnothing
    \label{eq:prod-deg-rxn}
\end{align}

The analytic solution to the associated master equation is (setting the system volume $\Omega$ equal to one):
\begin{equation}
    \Prob(x) = \sqrt{\frac{k_-}{2\pi k_+}}\exp\left( -\frac{k_-}{2k_+} \left( x - \frac{k_+}{k_-} \right)^2 \right)
    \label{eq:prod-deg-ans}
\end{equation}

I cite \cite{bistable-modeling}, \cite{we-exact}, \cite{we-chemkin}, and \cite{delay-oscillations} as well as \cite{gillespie-ssa}.

The IPython interactive interpreter and notebook (\cite{PER-GRA:2007}) has been of great use throughout this project.

\lipsum[3-4]

% section preview-material (end)

\bibliographystyle{apsrev}
\bibliography{citations}

\end{document}

