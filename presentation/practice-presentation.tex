\documentclass[xcolor={usenames,dvipsnames,svgnames}]{beamer}
\usepackage[utf8]{inputenc}
%\usepackage{caption}
\usepackage{graphicx}
\usepackage{subfigure}
\usepackage{multimedia}
\usepackage{hyperref}

%\AtBeginSection[] { \begin{frame} \frametitle{Table of Contents}
%\tableofcontents[currentsection] \end{frame} }
\usetheme{Berlin}
\usecolortheme{beaver}

% Custom math commands, other shortcuts
\newcommand{\tenexp}[1]{\times10^{#1}}
\newcommand{\dee}{\;\mathsf{d}}
\let\oldvec\vec
\renewcommand{\vec}[1]{\ensuremath{\mathbf{#1}}}
\newcommand{\evec}[1]{\ensuremath{\vec{e}_{#1}}} % standard basis vector
\newcommand{\norm}[2]{\ensuremath{\|#1\|_{#2}}}
\newcommand{\bignorm}[2]{\ensuremath{\left\|#1\right\|_{#2}}}
\newcommand{\infnorm}[1]{\ensuremath{\|#1\|_\infty}}
\newcommand{\reals}{\ensuremath{\mathbb{R}}}
\DeclareMathOperator{\Prob}{P}
% Physics Domain-Specific
\newcommand{\kB}{\ensuremath{k_\mathrm{B}}}
% General Shortcuts
\newcommand{\figref}[1]{Figure~\ref{#1}}
\newcommand{\secref}[1]{Section~\ref{#1}}

\begin{document}
\title[Biochemical Networks]{Stochastic Simulation of Biochemical Networks with Delays}
\author[Max Veit]{Max Veit\\Advisor: Jorge Viñals}
\date[2014-03-12]{12 March 2014}
\institute[U of MN]{University of Minnesota}
\subject{Physics}

\frame{\titlepage}

\begin{frame} \frametitle{Outline}
%    \tableofcontents[part=1,pausesections]
	\tableofcontents
\end{frame}

\section{Introduction} % (fold)
\label{sec:introduction}
\begin{frame}
	\frametitle{Biological Clocks}
	\begin{itemize}
            \item Periodic processes within cells
            \item Understood to be biochemically driven
            \item \textbf{TODO} Why are they important? What do we want to understand?
	\end{itemize}
\end{frame}

\begin{frame}
    \frametitle{Biochemical Networks}
    \begin{columns}[c]
        \begin{column}{0.5\textwidth}
            \begin{itemize}
                \item Systems of chemical reactions
                \item Reactions are interrelated
                \item Biological examples
                \item Understanding biological clocks as reaction networks
            \end{itemize}
        \end{column}
        \begin{column}{0.5\textwidth}
            \begin{align*}
                \varnothing \xrightarrow{A} X \\
                X \xrightarrow {B} \varnothing
            \end{align*}
        \end{column}
    \end{columns}
\end{frame}

\begin{frame}
    \frametitle{Chemical Kinetics}
    What aspect of these reaction networks is most important for studying biological clocks?
    \pause

    \emph{Time evolution.}
    \begin{itemize}
        \item How fast do the reactions run?
        \item Periodic, oscillatory behavior?
        \item Other behaviors (equilibrium, bistable, chaotic)?
        \item Bifurcations?
    \end{itemize}
\end{frame}

% section introduction (end)

\section{Modeling} % (fold)
\label{sec:modeling}

\begin{frame}
    \frametitle{Modeling}
    How do we model biochemical networks in order to study their behavior theoretically?
\end{frame}

\begin{frame}
    \frametitle{Reaction Rate Equations}
    \begin{itemize}
        \item Continuum limit
        \item Standard approach for large systems
        \item Turn chemical equations into differential equations
        \item Solve using standard techniques
    \end{itemize}
    \pause
    Example:
    \begin{columns}[c]
        \begin{column}{0.45\textwidth}
            \begin{align*}
                \varnothing \xrightarrow{A} X \\
                X \xrightarrow {B} \varnothing
            \end{align*}
        \end{column}
        \begin{column}{0.1\textwidth}
            \huge $\mapsto$
        \end{column}
        \begin{column}{0.45\textwidth}
            \begin{center}
                $\frac{\dee x(t)}{\dee t} = A - B x(t)$

                so

                $x(t) = \frac{A}{B} + x(0) e^{-B t}$
            \end{center}
        \end{column}
    \end{columns}
\end{frame}

\begin{frame}
    \frametitle{Why is a stochastic description necessary?}
    On a cellular scale, things look different:
    \hfill
    \begin{columns}[c]
        \begin{column}{0.5\textwidth}
            (Plot of concentration oscillating about a mean)
        \end{column}
        \begin{column}{0.5\textwidth}
            \begin{itemize}
                \item Concentrations oscillate about steady-state values
                \item Concentrations are discrete (``populations'')
                \item Noise is essential feature of some systems
                \item Continuum approximation can even give unphysical results
            \end{itemize}
        \end{column}
    \end{columns}
\end{frame}

\begin{frame}
    \frametitle{Stochastic Chemical Kinetics}
    Different approach: Describe the evolution of a \emph{probability distribution}
    \pause
    
    Time evolution described by Chemical Master equation (CME):
    \only<2>{
        \begin{multline*}
            \frac{\partial}{\partial t} P(\vec{x}, t | \vec{x_0}, t_0) = \\
            \sum_{j=1}^{M} \left( W_j(\vec{x} - \vec{s}_j) P(\vec{x} - \vec{s}_j, t | \vec{x}_0, t_0) - W_j(\vec{x}) P(\vec{x}, t | \vec{x}_0, t_0) \right)
        \end{multline*} 
    }
    \only<3>{
        \begin{multline*}
            \frac{\partial}{\partial t} P(\vec{x}, t | \vec{x_0}, t_0) = \\
            \sum_{{\color{BrickRed} j}=1}^{ {\color{BrickRed} M} } \left( W_{\color{BrickRed} j}(\vec{x} - \vec{s}_{\color{BrickRed} j}) P(\vec{x} - \vec{s}_{\color{BrickRed} j}, t | \vec{x}_0, t_0) - W_{\color{BrickRed} j}(\vec{x}) P(\vec{x}, t | \vec{x}_0, t_0) \right)
        \end{multline*} 
    }
    \only<4>{
        \begin{multline*}
            \frac{\partial}{\partial t} P(\vec{x}, t | \vec{x_0}, t_0) = \\
            \sum_{j=1}^{M} \left( {\color{NavyBlue} W_j}(\vec{x} - \vec{s}_j) P(\vec{x} - \vec{s}_j, t | \vec{x}_0, t_0) - {\color{NavyBlue}W_j}(\vec{x}) P(\vec{x}, t | \vec{x}_0, t_0) \right)
        \end{multline*} 
    }
    \only<5>{
        \begin{multline*}
            \frac{\partial}{\partial t} P(\vec{x}, t | \vec{x_0}, t_0) = \\
            \sum_{j=1}^{M} \left( {\color{NavyBlue} W_j}(\vec{x} - {\color{ForestGreen} \vec{s}_j}) P(\vec{x} - {\color{ForestGreen} \vec{s}_j}, t | \vec{x}_0, t_0) - {\color{NavyBlue}W_j}(\vec{x}) P(\vec{x}, t | \vec{x}_0, t_0) \right)
        \end{multline*} 
    }
    \only<3>{$\color{BrickRed} M$ reactions}
    \only<4-5>{Each reaction has a \emph{propensity} $\color{NavyBlue} W$}
    \only<5>{and a \emph{state change vector} $\color{ForestGreen} \vec{s}$}
\end{frame}

\begin{frame}
    \frametitle{Notes on Chemical Master Equation}
    \begin{itemize}
        \item Describes a \emph{Markovian} process\\
            (evolution depends only on current state)

        \item Assumptions:
        \begin{itemize}
            \item Well-stirred system
            \item Reactions occur instantaneously
            \item Ideal gas or dilute solution (!)
        \end{itemize}
    \end{itemize}
\end{frame}

% section modeling (end)

\section{Methodology} % (fold)
\label{sec:methodology}

% section methodology (end)

\section{Preliminary Results}

\begin{frame}[plain]

\hfill
    \begin{beamercolorbox}[rounded=true, center, shadow=true,wd=6cm]{title}
        \huge Questions?
    \end{beamercolorbox}
\hfill\hfill

\end{frame}

\end{document}
