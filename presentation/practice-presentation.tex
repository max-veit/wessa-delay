\documentclass[xcolor={usenames,dvipsnames,svgnames}]{beamer}
\usepackage[utf8]{inputenc}
\usepackage{graphicx}
\usepackage{subfigure}
\usepackage{hyperref}
\usepackage{mathtools}

%\AtBeginSection[] { \begin{frame} \frametitle{Table of Contents}
%\tableofcontents[currentsection] \end{frame} }
\usetheme{Berlin}
\usecolortheme{beaver}

% Custom math commands, other shortcuts
\newcommand{\tenexp}[1]{\times10^{#1}}
\newcommand{\dee}{\;\mathsf{d}}
\let\oldvec\vec
\renewcommand{\vec}[1]{\ensuremath{\mathbf{#1}}}
\newcommand{\evec}[1]{\ensuremath{\vec{e}_{#1}}} % standard basis vector
\newcommand{\norm}[2]{\ensuremath{\|#1\|_{#2}}}
\newcommand{\bignorm}[2]{\ensuremath{\left\|#1\right\|_{#2}}}
\newcommand{\infnorm}[1]{\ensuremath{\|#1\|_\infty}}
\newcommand{\reals}{\ensuremath{\mathbb{R}}}
\DeclareMathOperator{\Prob}{P}
% Physics Domain-Specific
\newcommand{\kB}{\ensuremath{k_\mathrm{B}}}
% General Shortcuts
\newcommand{\figref}[1]{Figure~\ref{#1}}
\newcommand{\secref}[1]{Section~\ref{#1}}

\begin{document}
\title[Biochemical Networks]{Stochastic Simulation of Biochemical Networks with Delays}
\author[Max Veit]{Max Veit\\Advisor: Jorge Viñals}
\date[2014-03-12]{12 March 2014}
\institute[U of MN]{University of Minnesota}
\subject{Physics}

\frame{\titlepage}

\begin{frame} \frametitle{Outline}
%    \tableofcontents[part=1,pausesections]
	\tableofcontents
\end{frame}

\section{Introduction} % (fold)
\label{sec:introduction}
\begin{frame}
	\frametitle{Biological Clocks}
	\begin{itemize}
            \item Periodic processes within cells
            \item Understood to be biochemically driven
            \item \textbf{TODO} Why are they important? What do we want to understand?
	\end{itemize}
\end{frame}

\begin{frame}
    \frametitle{Biochemical Networks}
    \begin{columns}[c]
        \begin{column}{0.5\textwidth}
            \begin{itemize}
                \item Systems of chemical reactions
                \item Reactions are interrelated
                \item Biological examples
                \item Understanding biological clocks as reaction networks
            \end{itemize}
        \end{column}
        \begin{column}{0.5\textwidth}
            \begin{align*}
                \varnothing \xrightarrow{A} X \\
                X \xrightarrow {B} \varnothing
            \end{align*}
        \end{column}
    \end{columns}
\end{frame}

\begin{frame}
    \frametitle{Chemical Kinetics}
    What aspect of these reaction networks is most important for studying biological clocks?
    \pause

    \emph{Time evolution.}
    \begin{itemize}
        \item How fast do the reactions run?
        \item Periodic, oscillatory behavior?
        \item Other behaviors (equilibrium, bistable, chaotic)?
        \item Bifurcations?
    \end{itemize}
\end{frame}

% section introduction (end)

\section{Modeling} % (fold)
\label{sec:modeling}

\begin{frame}
    \frametitle{Modeling}
    How do we model biochemical networks in order to study their behavior theoretically?
\end{frame}

\begin{frame}
    \frametitle{Reaction Rate Equations}
    \begin{itemize}
        \item Continuum limit
        \item Standard approach for large systems
        \item Turn chemical equations into differential equations
        \item Solve using standard techniques
    \end{itemize}
    \pause
    Example:
    \begin{columns}[c]
        \begin{column}{0.45\textwidth}
            \begin{align*}
                \varnothing \xrightarrow{A} X \\
                X \xrightarrow {B} \varnothing
            \end{align*}
        \end{column}
        \begin{column}{0.1\textwidth}
            \huge $\mapsto$
        \end{column}
        \begin{column}{0.45\textwidth}
            \begin{center}
                $\frac{\dee x(t)}{\dee t} = A - B x(t)$

                so

                $x(t) = \frac{A}{B} + x(0) e^{-B t}$
            \end{center}
        \end{column}
    \end{columns}
\end{frame}

\begin{frame}
    \frametitle{Why is a stochastic description necessary?}
    On a cellular scale, things look different:
    \hfill
    \begin{columns}[c]
        \begin{column}{0.5\textwidth}
            (Plot of concentration oscillating about a mean)
        \end{column}
        \begin{column}{0.5\textwidth}
            \begin{itemize}
                \item Concentrations oscillate about steady-state values
                \item Concentrations are discrete (``populations'')
                \item Noise is essential feature of some systems
                \item Continuum approximation can even give unphysical results
            \end{itemize}
        \end{column}
    \end{columns}
\end{frame}

\begin{frame}
    \frametitle{Stochastic Chemical Kinetics}
    Different approach: Describe the evolution of a \emph{probability distribution}.
    \pause
    
    Time evolution described by the chemical master equation (CME):
    \begin{overprint}
    \onslide<2>
        \begin{multline*}
            \frac{\partial}{\partial t} P(\vec{x}, t | \vec{x_0}, t_0) = \\
            \sum_{j=1}^{M} \left( a_j(\vec{x} - \vec{s}_j) P(\vec{x} - \vec{s}_j, t | \vec{x}_0, t_0) - a_j(\vec{x}) P(\vec{x}, t | \vec{x}_0, t_0) \right)
        \end{multline*} 
    \onslide<3>
        \begin{multline*}
            \frac{\partial}{\partial t} P(\vec{x}, t | \vec{x_0}, t_0) = \\
            \sum_{{\color{BrickRed} j}=1}^{ {\color{BrickRed} M} } \left( a_{\color{BrickRed} j}(\vec{x} - \vec{s}_{\color{BrickRed} j}) P(\vec{x} - \vec{s}_{\color{BrickRed} j}, t | \vec{x}_0, t_0) - a_{\color{BrickRed} j}(\vec{x}) P(\vec{x}, t | \vec{x}_0, t_0) \right)
        \end{multline*} 

        $\color{BrickRed} M$ reactions
    \onslide<4>
        \begin{multline*}
            \frac{\partial}{\partial t} P(\vec{x}, t | \vec{x_0}, t_0) = \\
            \sum_{j=1}^{M} \left( {\color{NavyBlue} a_j}(\vec{x} - \vec{s}_j) P(\vec{x} - \vec{s}_j, t | \vec{x}_0, t_0) - {\color{NavyBlue}a_j}(\vec{x}) P(\vec{x}, t | \vec{x}_0, t_0) \right)
        \end{multline*} 

        Each reaction has a \emph{propensity} $\color{NavyBlue} a$
    \onslide<5>
        \begin{multline*}
            \frac{\partial}{\partial t} P(\vec{x}, t | \vec{x_0}, t_0) = \\
            \sum_{j=1}^{M} \left( {\color{NavyBlue} a_j}(\vec{x} - {\color{ForestGreen} \vec{s}_j}) P(\vec{x} - {\color{ForestGreen} \vec{s}_j}, t | \vec{x}_0, t_0) - {\color{NavyBlue}a_j}(\vec{x}) P(\vec{x}, t | \vec{x}_0, t_0) \right)
        \end{multline*} 

        Each reaction has a \emph{propensity} $\color{NavyBlue} a$ and a \emph{state change vector} $\color{ForestGreen} \vec{s}$.
    \end{overprint}
\end{frame}

\begin{frame}
    \frametitle{Notes on Chemical Master Equation}
    \begin{itemize}
        \item Describes a \emph{Markovian} process\\
            (evolution depends only on current state)

        \item Assumptions about system:
        \begin{itemize}
            \item Reactions occur instantaneously
            \item Well-stirred system
            \item Ideal gas or dilute solution (!)
        \end{itemize}
    \end{itemize}
\end{frame}

\begin{frame}
    \frametitle{Abstraction: Delayed Reactions}
    \begin{itemize}
        \item Biological processes consist of many small reactions
        \item Often don't want to simulate every step
        \item Abstract sequence into one reaction with a delay
    \end{itemize}
    (Figure: Delayed reactions)
    \pause

    \emph{Note}: delay breaks Markovian property!
\end{frame}

% section modeling (end)

\section{Methodology} % (fold)
\label{sec:methodology}

\begin{frame}
    \frametitle{Methodology}
    How do we solve the chemical master equation to obtain the time evolution of a stochastic system?
\end{frame}

\begin{frame}
    \frametitle{Stochastic Simulation Algorithm (SSA)}
    Introduced by Daniel Gillespie, 1976.

    Features:
    \begin{itemize}
        \item Monte Carlo approach
        \item Generates a \emph{sample} (``trajectory'')\\
            of the probability distribution
        \item Simulates a sequence of reactions
        \begin{itemize}
            \item Reaction type and time\\
                randomly chosen using propensities
            \item Wait time is exponentially distributed\\
                (Markovian property)
        \end{itemize}
    \end{itemize}
    Typically use an ensemble of trajectories.
\end{frame}

\begin{frame}
    %TODO Call this the ``concentration space'' instead?
    \frametitle{Undersampling the Phase Space}
    Trajectories spend lots of time in certain regions of space.

    Other regions get ignored.

    (Figure: Sampling a bimodal distribution)
\end{frame}

\begin{frame}
    \frametitle{Resampling Using a Weighted Ensemble}
    Idea: Assign each trajectory in the ensembe a weight $w_j$.\\
    Divide the phase space into bins.
    \begin{columns}[c]
        \begin{column}{0.5\textwidth}
            \begin{enumerate}
                \item Run trajectories for time $t_s$.
                \item Pause trajectories.\\
                    In each bin:
                \pause
                \begin{itemize}
                    \item If not enough trajectories,\\
                        split some.
                    \pause
                    \item If too many,\\
                        combine some.
                    \pause
                    \item In any case,\\
                        \emph{conserve weight} in bin.
                \end{itemize}
                \pause
                \item Repeat.
            \end{enumerate}
            \pause
            Statistical \emph{resampling} procedure.
        \end{column}
        \begin{column}{0.5\textwidth}
            (Sequence of images to illustrate weighted ensemble)
        \end{column}
    \end{columns}
\end{frame}

\begin{frame}
    \frametitle{Conceptual Issues}
    SSA does not use \emph{uniform} timesteps. How to pause?

    (Figure illustrating non-uniform timesteps)

    Not a problem \emph{if} system is Markovian.
\end{frame}

\begin{frame}
    \frametitle{Delays Complicate Things Further}
    (Figure illustrating delayed reactions)

    From where is delay measured?
\end{frame}

\begin{frame}
    \frametitle{(Partial) Resolution}
    \begin{itemize}
        \item Not \emph{inaccurate} resampling, just suboptimal\ldots
        \item Physical standpoint
        \item Trajectory vs. system
    \end{itemize}
\end{frame}

\begin{frame}
    \frametitle{Numerical Evidence on Resampling}
    (Plot of Gaussian production-degradation system)
\end{frame}

% section methodology (end)

\section{Preliminary Results}

\begin{frame}
    \frametitle{Delayed-Degradation System}
    \begin{columns}[c]
        \begin{column}{0.5\textwidth}
            \begin{align*}
                \varnothing \xrightarrow{A} X \\
                X \xrightarrow{B} \varnothing \\
                X \xRightarrow[(\tau)]{C} \varnothing
            \end{align*}

            Common model system
        \end{column}
        \begin{column}{0.5\textwidth}
            (Plot of DPD time evln)
        \end{column}
    \end{columns}
\end{frame}

%TODO More complex biological clocks: If plot available, show it.

\begin{frame}
    \frametitle{Upcoming Goals: Modeling Cells}
    Cells don't meet assumptions used by the SSA.

    Relax some assumptions:
    \begin{itemize}
        \item Crowded environment: modify reaction rates
        \item Inhomogeneity: introduce diffusion
    \end{itemize}
\end{frame}

\begin{frame}
    \frametitle{Summary and Project Goals}
    \begin{itemize}
        \item Adapt weighted-ensemble SSA for delays
        \item Apply simulation to common model systems
        \begin{itemize}
            \item Study system properties
            \item Validate/challenge common assumptions
        \end{itemize}
        \item Use simulation on more complex systems
        \item More realistically model cells
    \end{itemize}
\end{frame}

\begin{frame}[plain]

\hfill
    \begin{beamercolorbox}[rounded=true, center, shadow=true,wd=6cm]{title}
        \huge Questions?
    \end{beamercolorbox}
\hfill\hfill

\end{frame}

\end{document}
